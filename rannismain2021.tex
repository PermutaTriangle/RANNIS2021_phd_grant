% -*- coding: utf-8 -*-
\documentclass{rannis}

\usepackage{xargs}
\usepackage{ifthen}
% chktex-file 1
% chktex-file 8
% chktex-file 31
% chktex-file 36
%%%%%%%%%%% MACROS FOR DRAWING INTERVAL AND MESH PATTERNS %%%%%%%%%%%

% Sub-macros
\newcommand{\shadetheboxes}[1]{
    \foreach \x/\y in {#1}
    \fill[pattern color = black!75, pattern=north east lines] (\x,\y) rectangle +(1,1);
}

\newcommand{\drawthegrid}[1]{
    \draw (0.01,0.01) grid (#1+0.99,#1+0.99);
}

\newcommand{\drawverticallines}[3]{
    \foreach \x in {#2}
    \draw[line width=#3] (\x+0.01,0.01) -- (\x+0.01,#1+0.99);
}

\newcommand{\drawhorizontallines}[3]{
    \foreach \y in {#2}
    \draw[line width=#3] (0.01,\y+0.01) -- (#1+0.99,\y+0.01);
}

\newcommand{\drawtheclpattern}[1]{
    \foreach \x/\y in {#1}
    \filldraw (\x,\y) circle (6pt);
}

\newcommand{\drawclpattern}[2]{
	\foreach[count=\x] \y in {#1}
	{
		\filldraw (\x,\y) circle (#2 pt);
	}
}

\newcommand{\drawspecialbox}[1]{
    \foreach \x/\y/\z/\w/\A in {#1}
    {
        \fill[color = white!100, opacity=1, rounded corners = 1.5pt] (\x+0.125,\y+0.125) rectangle (\z-0.125,\w-0.125);
        \draw[color = black, rounded corners = 1.5pt] (\x+0.125,\y+0.125) rectangle (\z-0.125,\w-0.125);
        \fill[black] (\x/2+\z/2,\y/2+\w/2) node {\A};
    }
}

\newcommand{\drawspecialboxlarge}[1]{
    \foreach \x/\y/\z/\w/\A in {#1}
    {
        \fill[color = white!100, opacity=1, rounded corners = 1.5pt] (\x+0.125,\y+0.125) rectangle (\z-0.125,\w-0.125);
        \draw[color = black, rounded corners = 1.5pt] (\x+0.125,\y+0.125) rectangle (\z-0.125,\w-0.125);
        \fill[black] (\x/2+\z/2,\y/2+\w/2) node {\Large \A};
    }
}

\newcommand{\drawsolidshadedbox}[1]{
    \foreach \x/\y/\z/\w/\A in {#1}
    {
        \fill[color = gray!50, opacity=1, rounded corners=1.5pt] (\x+0.125,\y+0.125) rectangle (\z-0.125,\w-0.125);
        \draw[color = black, rounded corners=1.5pt] (\x+0.125,\y+0.125) rectangle (\z-0.125,\w-0.125);
        \fill[black] (\x/2+\z/2,\y/2+\w/2) node {\A};
    }
}

\newcommand{\drawlabels}[1]{
	\foreach \x/\y/\lab in {#1}
	{
		\draw (\x + 0.5,\y + 0.5) node {\lab};
	}
}


\newcommand{\pOneTwo}[1]{\mbox{\patt{#1}{2}{1,2}[][][][][][7]}}
\newcommand{\pTwoOne}[1]{\mbox{\patt{#1}{2}{2,1}[][][][][][7]}}
\newcommand{\pOneTwoThree}[1]{\mbox{\patt{#1}{3}{1,2,3}[][][][][][7]}}
\newcommand{\pOneThreeTwo}[1]{\mbox{\patt{#1}{3}{1,3,2}[][][][][][7]}}
\newcommand{\pTwoOneThree}[1]{\mbox{\patt{#1}{3}{2,1,3}[][][][][][7]}}
\newcommand{\pOneThreeTwoFour}[1]{\mbox{\patt{#1}{4}{1,3,2,4}[][][][][][7]}}

\newcommand{\etcdots}[2]{
	\scalebox{#1}
	{
		\begin{tikzpicture}[baseline=(current bounding box.center)]
			\filldraw (0,2) circle (#2 pt);
			\filldraw (1,1) circle (#2 pt);
			\filldraw (2,0) circle (#2 pt);
		\end{tikzpicture}
	}
}

\newcommand{\etcdotsflipped}[2]{
    \scalebox{#1}
    {
        \begin{tikzpicture}[baseline=(current bounding box.center)]
            \filldraw (0,0) circle (#2 pt);
            \filldraw (1,1) circle (#2 pt);
            \filldraw (2,2) circle (#2 pt);
        \end{tikzpicture}
    }
}

\newcommand{\decr}{\etcdots{0.2}{6}}
\newcommand{\incr}{\etcdotsflipped{0.2}{6}}

% \[
%   \metapatt{0.6}{1/2}{1/v,1/h,2/h}{}{1/2/p}[0/1,0/2,1/0,1/2][0/0/1/1/$\decr$,1/1/2/2/$\incr$][]
% \]

% #1: Scale
% #2: Length
% #3: Points
% #4: Shades
% #5: Markings
% #6: Avoidance decorations
% #7: Containment decorations
% #8: Labels
% #9: Size of the points
\newcommandx{\patt}[9][4={},5={},6={},7={},8={},9=4]
{
	\scalebox{#1}
	{
		\begin{tikzpicture}[baseline=(current bounding box.center)]
			\useasboundingbox (0.0,-.3) rectangle (#2+1,#2+1.3);
			\shadetheboxes{#4}
			\draw (0.01,0.01) grid (#2+1-0.01,#2+1-0.01);

			\drawsolidshadedbox{#6}
			\drawspecialbox{#7}
			\drawspecialboxlarge{#5}
			\drawclpattern{#3}{#9}
			\drawlabels{#8}
		\end{tikzpicture}
	}
}

% #1: Scale
% #2: Length
% #3: Points
% #4: Shades
% #5: Markings
% #6: Avoidance decorations
% #7: Containment decorations
% #8: Circled points
\newcommandx{\cpatt}[8][4={},5={},6={},7={},8={}]
{
	\scalebox{#1}
	{
		\begin{tikzpicture}[baseline=(current bounding box.center)]
			\useasboundingbox (0.0,-.3) rectangle (#2+1,#2+1.3);
			\shadetheboxes{#4}
			\draw (0.01,0.01) grid (#2+1-0.01,#2+1-0.01);

			\drawsolidshadedbox{#6}
			\drawspecialbox{#7}
			\drawspecialboxlarge{#5}
			\drawclpattern{#3}{4}

			\foreach \x/\y in {#8}
			{
				\draw[line width=1] (\x,\y) circle (7 pt);
			}
		\end{tikzpicture}
	}
}


% #1: Scale
% #2: Width/Height
% #3: Pattern lines
% #4: Added lines
% #5: Points
% #6: Shadings
% #7: Markings
% #8: Ascending restrictions
\newcommandx{\metapatt}[8][6={},7={},8={}]
{
    \scalebox{#1}
    {
        \begin{tikzpicture}[baseline={([yshift=-6pt]current bounding box.center)}]
					\foreach \width/\height in {#2}
					{
						\useasboundingbox (0.0,-.3) rectangle (\width+1,\height+1.3);
            \shadetheboxes{#6}

            \foreach \pos/\type in {#4}
            {
                \ifthenelse{\equal{\type}{v}}
                {
                    \drawverticallines{\height}{\pos}{1.7pt}
                }
                {
								    \ifthenelse{\equal{\type}{d}}
                    {
                      \draw[densely dashed] (\pos,0) -- (\pos,\height+1);
                    }
										{
											\drawhorizontallines{\width}{\pos}{1.7pt}
										}
                }
            }

            \foreach \pos/\type in {#3}
            {
                \ifthenelse{\equal{\type}{v}}
                {
                    \drawverticallines{\height}{\pos}{0.6pt}
                }
                {
										\drawhorizontallines{\width}{\pos}{0.6pt}
                }
            }

            \drawsolidshadedbox{#8}
            \drawspecialbox{#7}

            \foreach \x/\y/\type in {#5}
            {
                \ifthenelse{\equal{\type}{a}}
                {
                    % Added point
                    \draw (\x,\y) circle (6pt);
                    \filldraw (\x,\y) circle (3pt);
                }
                {
                    % Pattern point
                    \filldraw (\x,\y) circle (4pt);
                }
            }
					}
        \end{tikzpicture}
    }
}

% #1: Scale
% #2: Width/Height
% #3: Pattern lines
% #4: Added lines
% #5: Points
% #6: Shadings
% #7: Markings
% #8: Ascending restrictions
% #9: Labels
\newcommandx{\dpatt}[9][6={},7={},8={},9={}]
{
    \scalebox{#1}
    {
        \begin{tikzpicture}[baseline=(current bounding box.center)]
					\foreach \width/\height in {#2}
					{
						\useasboundingbox (0.0,-.3) rectangle (\width+1,\height+1.3);
            \shadetheboxes{#6}

            \foreach \pos/\type in {#4}
            {
                \ifthenelse{\equal{\type}{v}}
                {
                    \drawverticallines{\height}{\pos}{1.7pt}
                }
                {
								    \ifthenelse{\equal{\type}{d}}
                    {
                      \draw[densely dashed] (\pos,0) -- (\pos,\height+1);
                    }
										{
											\drawhorizontallines{\width}{\pos}{1.7pt}
										}
                }
            }

            \foreach \pos/\type in {#3}
            {
                \ifthenelse{\equal{\type}{v}}
                {
                    \drawverticallines{\height}{\pos}{0.6pt}
                }
                {
										\drawhorizontallines{\width}{\pos}{0.6pt}
                }
            }

            \drawsolidshadedbox{#8}
            \drawspecialbox{#7}

            \foreach \x/\y/\type in {#5}
            {
                \ifthenelse{\equal{\type}{a}}
                {
                    % Added point
                    \draw (\x,\y) circle (6pt);
                    \filldraw (\x,\y) circle (3pt);
                }
                {
                    % Pattern point
                    \filldraw (\x,\y) circle (4pt);
                }
            }

						\drawlabels{#9}
					}
        \end{tikzpicture}
    }
}


\usepackage{todonotes}

\usepackage{amsmath}
\usepackage{amsthm}
\usepackage{subcaption}

\newcommand{\motheralg}{\textsf{Combex}}
\newcommand{\explofmothername}{\underline{Comb}inatorial \underline{Ex}ploration}
\newcommand{\tilealg}{\textsf{PermScope}}
\newcommand{\partialg}{\textsf{PartiScope}}

\newcommand{\bisc}{\textsf{BiSC}}
\newcommand{\struct}{\textsf{Struct}}

\newcommand{\yearofappl}{2021}
\newcommand{\yearofapplp}{2022}
\newcommand{\yearofapplpp}{2023}

\newcommand{\yearofappls}{21}
\newcommand{\yearofapplsp}{22}
\newcommand{\yearofapplspp}{23}

\newcommand{\disjoint}{\sqcup}
\newcommand{\eqg}{\cong}

% mathcal letters
\newcommand{\mc}[1]{\mathcal{#1}}
\newcommand{\mb}[1]{\mathbb{#1}}

\newcommand{\cA}{{\mathcal{A}}}
\newcommand{\cB}{{\mathcal{B}}}
\newcommand{\cC}{{\mathcal{C}}}
\newcommand{\cD}{{\mathcal{D}}}
\newcommand{\cE}{{\mathcal{E}}}
\newcommand{\cI}{{\mathcal{I}}}
\newcommand{\cS}{{\mathcal{S}}}
\newcommand{\perms}{\mathcal{S}}
\newcommand{\gclass}{\mathcal{G}}
\newcommand{\Av}{\mathrm{Av}}
\newcommand{\Co}{\mathrm{Co}}

% For tikz pictures
\usepackage{tikz}
\usetikzlibrary{matrix,arrows}
\usetikzlibrary{positioning}
\usetikzlibrary{fit}
\usetikzlibrary{patterns}

\newcommand{\imopattern}[6]{ % mesh pattern with white and black dots overlapping
  \raisebox{0.6ex}{
  \begin{tikzpicture}[scale=0.35, baseline=(current bounding box.center), #1]
    \foreach \x/\y in {#6}
      \fill[pattern=north east lines] (\x,\y) rectangle +(1,1);
    \draw (0.01,0.01) grid (#2+0.99,#2+0.99);
    \foreach \x/\y in {#4}
      \draw[fill=white] (\x,\y) circle (6pt);
    \foreach \x/\y in {#5}
      \draw[fill=white] (\x,\y) circle (10pt);
    \foreach \x/\y in {#3}
      \filldraw (\x,\y) circle (6pt);
  \end{tikzpicture}
  }
}

% Some environments
\newtheorem{theorem}{Theorem}
\newtheorem{conjecture}[theorem]{Conjecture}
\newtheorem*{conjecture*}{Conjecture}

\theoremstyle{definition}
\newtheorem{example}[theorem]{Example}
\newtheorem{definition}[theorem]{Definition}

\newenvironment{packed_item}{
\begin{itemize}
  \setlength{\itemsep}{-1pt}
  \setlength{\parskip}{-1pt}
  \setlength{\parsep}{-1pt}
}{\end{itemize}}

\newenvironment{packed_enum}{
\begin{enumerate}
  \setlength{\itemsep}{-1pt}
  \setlength{\parskip}{-1pt}
  \setlength{\parsep}{-1pt}
}{\end{enumerate}}

\usepackage[utf8]{inputenc}

% Language and input.
\usepackage[icelandic,british]{babel}
\usepackage{wrapfig}
\usepackage{graphicx}

\title{Extending the Combex framework}
\author{\'Emile Nadeau}

\begin{document}

\newcommand{\lastpageref}{\pageref{LastPageOfDescription}}
\thispagestyle{empty}

\maketitle

\newpage

\section*{Guidelines}\label{guidelines}

\noindent
Please note that \emph{all proposals and appendices} should be in
English.\\

\noindent
The text in angle brackets (<text>) in this document is provided for guidance
purposes only, and should be overwritten, by applicant, with the \mbox{requested information}.
Please insert appropriate material (i.e.\ text, pictures and/or tables)
and add the principal investigator's name to the header and the
project's name to the footer.\\

\noindent
The detailed project description should provide the following
information, and be divided into the following sections (the order and
titles should not be changed):\\
{\it
\begin{enumerate}
\def\labelenumi{\Alph{enumi}.}
\setlength\itemsep{-0.5em}
\item
  \begin{quote}
  Specific aims of the project, research questions/hypotheses,
  feasibility, originality and impact
  \end{quote}
\item
  \begin{quote}
  Present state of knowledge in the field
  \end{quote}
\item
  \begin{quote}
  Research plan (time and work plan, present status of project,
  methodology and milestones) and deliverables
  \end{quote}
\item
  \begin{quote}
  Management and co-operation (domestic/foreign)
  \end{quote}
\item
  \begin{quote}
  Proposed publication of results and data storage (including open
  access policy)
  \end{quote}
\item
  \begin{quote}
  Contribution of doctoral and master's degree students to the
  project\\
  \end{quote}
\end{enumerate}
}

\noindent
The project description should not exceed \textbf{5 pages} plus
front page and guidelines (1.5 line spacing, 12 point Times/Times New
Roman, or similar, 2.5 cm side margins). The \mbox{bibliography} is submitted
in a separate file with no length limits. Convert the file to pdf before submitting. Please note:

\begin{itemize}
\setlength\itemsep{-0.5em}
\item[$\rightarrow$]
  \emph{\textbf{Incomplete proposals will be rejected}}
\item[$\rightarrow$]
  \emph{\textbf{Proposals which exceed length limitations will be
  rejected without review}}
\item[$\rightarrow$]
  \emph{\textbf{Corrections or amendments after the previously announced
  deadline will be rejected}}
\item[$\rightarrow$]
  \emph{\textbf{The clarity and the overall quality of the presentation
  are taken into consideration when proposals are reviewed}}
\item[$\rightarrow$]
  \emph{\textbf{Applications that are not submitted using the most
  recent application form available will be rejected}}\\
\end{itemize}

\noindent
For further information please refer to the \emph{\textbf{Icelandic
Research Fund Handbook 2021.}}\\

\emph{\textbf{Please do not delete, overwrite, or amend this Guideline
page in your submission\\
}}

\newpage

\spacing{1}
\section{Specific aims of the project, research questions/hypotheses, feasibility, originality and impact}
\spacing{1.5}
The goal of this proposal is to extend an existing framework, developed at
Reykjavik University, which
leverages domain-specific knowledge to discover and automatically prove
theorems in mathematics.
\emph{Combinatorial exploration} is an experimental approach that
rigorously derives structural results about mathematical objects.
When a human has discovered the structure of an object there
are several tools, often automatic, which allow various properties of the object
to be computed. However, the steps from the problem statement and the original
object to the structure is often ad-hoc. This is the gap the framework fills.
A beta version of the framework, called
\motheralg{} (\explofmothername) exists. It has mainly been applied in
the field of permutation patterns and is called in that context the
\tilealg. This algorithm has been able
to discover new theorems and rediscover several results
spanning dozens of papers in the research literature.


We propose to integrate techniques from machine learning to enhance the beta,
as well as improving the algorithm to find more general descriptions
of mathematical objects.
The outcome of this proposal will be
publications in journals and presentations at international
conferences. The implementation of
these theoretical algorithms will be made available free and open source and
will be accompanied by a comprehensive manual.
\subsection*{The \motheralg{} framework}

\motheralg{} performs combinatorial
exploration in two main stages. First is the expansion stage, where \emph{strategies} are
applied to the objects of interest, resulting in a set of rules
relating the objects. The second stage is to check if there exists
a \emph{specification} for the original object of interest
within this set of rules. The term \emph{specification} has a technical meaning,
but we can't think of it as a system of equations describing the structure of an object.

After a predefined interval in the expansion stage of combinatorial exploration, \motheralg{} enters a searching stage. The expansion stage will have produced a \textit{universe} of combinatorial rules. \motheralg{} now searches this universe for specifications, and is guaranteed to find them if they exist. If no specifications are found, \motheralg{} re-enters the expansion stage.

We will show below how \motheralg{} has been applied to the field of permutation
patterns.
As is to be expected with a beta version there is room for improvement. In
particular, even though the
framework is guaranteed to find a specification in the universe if
one exists, there are cases where the universe contains enough information to
describe the original object, without any specification existing for it.
In this case we say that the universe contains a \emph{combinatorial system} for
the object.
Even if some combinatorial systems have been found by hand \cite{BeanPhd},
there is currently no way to find such system automatically.
The proposer will develop heuristics and algorithms to find combinatorial systems.

\subsection*{The \tilealg{} algorithm}
A \emph{permutation} is a reordering of the integers $1, 2, \dotsc, n$
for some $n$. Although
these are simple objects, they have deep connections to other fields. Famous
examples include Schubert varieties~\cite{MR1051089}, posets~\cite{MR2652101},
water waves~\cite{MR2813307}, genome rearrangements~\cite{MR2518996}, planar maps~\cite{CKS09}
meanders~\cite{fukuda:042202}, and sorting operators~\cite{MR0445948}.
A permutation $\pi$ \emph{contains} the permutation $p$ if certain technical
conditions are satisfied. We only give an example here: The permutation $\pi = 526413$
contains $p = 132$,  in the subsequences $264$, $263$ and $243$; see
Figure~\ref{fig:perm526413} where we have additionally circled the occurrences of $p$.

\begin{figure}[htbp]
\begin{center}
\imopattern{scale=0.7}{ 6 }{ 1/5, 2/2, 3/6, 4/4, 5/1, 6/3 }
{}{ 2/2, 3/6, 4/4 }{}\qquad
\imopattern{scale=0.7}{ 6 }{ 1/5, 2/2, 3/6, 4/4, 5/1, 6/3 }
{}{ 2/2, 3/6, 6/3 }{}\qquad
\imopattern{scale=0.7}{ 6 }{ 1/5, 2/2, 3/6, 4/4, 5/1, 6/3 }
{}{ 2/2, 4/4, 6/3 }{}
\caption{The permutation $526413$ and three occurrences of the $132$}%
\label{fig:perm526413}
\end{center}
\end{figure}

\noindent
The same permutation \emph{avoids} $123$, as it has no
increasing subsequence of length three.  In this context we refer to $p$ as a
 \emph{pattern}. The main object of study in this field is
the (infinite) set of permutations that avoid a given pattern, denoted $\Av(p)$,
and called a \emph{permutation class}.

Domain-specific knowledge from the field of permutation
patterns was added to \motheralg{} to create the
\tilealg{} algorithm.
When \tilealg{} succeeds it produces a specification in the form of a tree,
shown in Figure~\ref{fig:123_1432}, for the permutations avoiding the patterns $123$
and $1432$, first enumerated by West~\cite{West:1996cy}.
\begin{figure}[ht]
    \centering
        \begin{tikzpicture}[scale = 0.7, vertex/.style = {rectangle,draw = black!50,fill = gray!5,thick}, auto]

         \node [vertex] (root)      at (0,0)     {$\cB$};
         \node [vertex] (empty)  at (2,-2)   {$\cE$};
         \node [vertex] (nempty) at (2,0)    {$\metapatt{0.4}{1/1}{1/v,1/h}{}{1/1/p}[0/1,1/1][0/0/1/1/$\cD$,1/0/2/1/$\cB$]$};
         \node [vertex] (nemptyl) at (5,0)    {$\metapatt{0.4}{1/1}{1/v,1/h}{}{1/1/p}[0/1,1/1,0/0][1/0/2/1/$\cB$]$};
         \node [vertex] (nemptyr) at (5,-2.5)    {$\metapatt{0.4}{2/2}{1/v,2/v,1/h,2/h}{}{1/1/p,2/2/p}[0/2,1/2,2/2,0/0,1/0,1/1][0/1/1/2/$\cD$,2/1/3/2/$\cD$,2/0/3/1/$\cB$]$};
         \node [vertex] (nemptyrl) at (9,-3)    {$\metapatt{0.4}{2/2}{1/v,2/v,1/h,2/h}{}{1/1/p,2/2/p}[0/2,1/2,2/2,0/0,1/0,1/1,2/1][0/1/1/2/$\cD$,2/0/3/1/$\cB$]$};
         \node [vertex] (nemptyrr) at (9,0)    {$\metapatt{0.4}{4/3}{1/v,2/v,3/v,4/v,1/h,2/h,3/h}{}{2/1/p,3/3/p,4/2/p}[0/3,1/3,2/3,3/3,0/0,1/0,0/1,1/2,2/1,2/2,3/1,3/2,2/0,4/1,4/2,4/3][1/1/2/2/$\cD$,0/2/1/3/$\cD$,3/0/4/1/$\cD$,4/0/5/1/$\cB$]$};
         \node [vertex] (factorl) at (13,.2)    {$\metapatt{0.4}{3/3}{1/v,2/v,3/v,1/h,2/h,3/h}{}{2/1/p,3/3/p}[0/3,1/3,2/3,3/3,0/0,1/0,0/1,1/2,2/1,2/2,3/0,3/1,3/2,2/0][1/1/2/2/$\cD$,0/2/1/3/$\cD$]$};
         \node [vertex] (factorr) at (13.5,-2.5)    {$\metapatt{0.4}{1/1}{1/v,1/h}{}{1/1/p}[0/1,1/1][0/0/1/1/$\cD$,1/0/2/1/$\cB$]$};


         \draw [-,semithick] (root) to (empty);
         \draw [-,semithick] (root) to (nempty);
         \draw [-,semithick] (nempty) to (nemptyl);
         \draw [-,semithick] (nempty) to (nemptyr);
         \draw [-,semithick] (nemptyr) to (nemptyrl);
         \draw [-,semithick] (nemptyr) to (nemptyrr);
         \draw [-,semithick] (nemptyrr) to (factorl);
         \draw [-,semithick] (nemptyrr) to (factorr);

        \end{tikzpicture}
    \caption{The structure of $\Av(123, 1432)$ in terms of simpler substructures}%
    \label{fig:123_1432}
\end{figure}
So far, the \tilealg{} as proved to be a excellent
tool,  providing structural and enumerative results that together subsume over
a dozen of existing research papers.

The proposer will undertake the creation
a generalization of the \tilealg{} that describes the structure of permutation
classes in a finer way by accounting for permutation statistics.
Permutation statistics which are special function from the set of
permutations to $\mathbb{N}$
are an active area of research in permutation patterns
\cite{bukata2019statisitics,dokos2012statistics,
babson2000vincular,branden2011meshpattern}.
One example of permutation statistics is the \emph{inversion count} that we
briefly introduce.
An \emph{inversion} in a permutation $\pi$ is a pair of integer $x < y$ such that $y$
comes before $x$ in $\pi$. The inversion of the permutation $526413$
(see Figure~\ref{fig:perm526413}) are $52, 54,51,53,21,64,61,63,41$ and $43$.
The inversion count of $\pi$, denoted $\mathrm{inv}(\pi)$, is 9.


\subsection*{Machine learning and heuristics}
The \motheralg{} beta is a naive breadth-first search algorithm. The
general objectives which the algorithm is trying to accomplish have clear
analogies with other tree-search paradigms. For instance
\emph{iterative deepening}, i.e., pursuing a search deeper at promising nodes
before returning to a generally breadth-first approach. Since there is no
existing expert knowledge in the field to determine how to evaluate what
constitutes a ``promising'' node this approach would need to be coupled with
machine learning of such evaluation functions based on the beta's
experience over a wide variety of training examples, both successful and
unsuccessful.

\subsection*{Feasibility, originality, and impact}
Given a
specification for a class there are methods, often automatic, for
obtaining a system of equations and finding the generating function.
This can then be used to give the enumerative information for
the class and many other properties. To find a
specification for a class, mathematicians have often
used ad-hoc techniques and the theory has not been developed as much.
\motheralg{} and the theory of combinatorial exploration will bridge this gap
and automate the process of going from a class to a
specification. Rather than rediscovering similar methods and repeating the same
ideas, \motheralg{} will push mathematical developments in new directions.

The beta version has been very successful and found
specifications for classes that had entire papers devoted to them, including but
not limited to~\cite{inflationsofgridclasses, smoothclasses, le-wilfclasses,
dbis, jaysgridclasses}.
Despite these successes, numerous interesting cases still remain, in particular,
$\Av(1324)$, the biggest open problem in the field, against which every new
proof method is tested.
If we are successful in finding a structure
and generating function for unenumerated classes, in particular the
$1324$-avoiders, it would be considered a major breakthrough. Even tighter bounds
on the enumeration of this set are worthy of publication.
\\

\spacing{1}
\section{Present state of knowledge in the field}
\spacing{1.5}
There exist specialized algorithms for enumerating permutation classes, such as
enumeration schemes \mbox{Zeilberger}~\cite{MR1682929} with a generalization
that accounts for statistics \mbox{Baxter}~\cite{baxter2014scheme_statistics};
regular insertion encoding \mbox{Albert}, \mbox{Linton}, and \mbox{Ru\v{s}kuc}~\cite{MR2176523};
substitution decomposition of classes with finitely many simple permutations \mbox{Bassino}
et al.~\cite{subdecompimplement}; and
guessing a disjoint union of grid classes Bean, Gudmundsson, and Ulfarsson~\cite{algstruct, structpaper}.
These algorithms are deterministic and either succeed or fail on their inputs.
The \motheralg{} framework is different in that it generates a universe of true
statements about the object under investigation, and from this universe one can
often extract enough information to reveal the structure of the object. What we
are proposing in this application is strengthening the framework in such a way that
it can reveal the structure of the object with a smaller universe, therefore making
it applicable to more problems.
\\

\spacing{1}
\section{Research plan (time and work plan, methodology, milestones, present status of project, etc.) and deliverables. Refer to more detailed description of milestones and deliverables in part  3 in the IRF electronic application system. Explain if consents and/or permits are needed}
\spacing{1.5}

\noindent \textbf{Time and work plan.}
The work packets are detailed under milestones below, as well as when they will
be attempted.

\noindent\textbf{Methodology.}
Some methods used will be traditional, coming from combinatorics, algorithms and
computer science. We will also use newer tools, such as generalized grid classes
and mesh patterns. We will also introduce new strategies, and
apply methods from computer science, such as probabilistic search and machine
learning.

\noindent \textbf{Milestones.}
\begin{packed_enum}

\item[WP1] Develop algorithm to automatically finds combinatorial systems with
    a single variable automatically.  (Jan-Jun '\yearofappls)
\item[WP2] Develop algorithm to automatically finds combinatorial systems
    in many variables automatically.  (Jul-Dec '\yearofappls)
\item[WP3] Implement iterative deepening search in the framework. (Jan-Jun '\yearofapplsp)
\item[WP4] Extend the \tilealg{} algorithm to find specification and
    combinatorial system that describe permutation statistics in classes.
    (Jul-Dec '\yearofapplsp)
    % Name in system: C1 Iterative deepening
    % Start: 25
    % End: 36
    % Project: Phd1 (EN)
    % Milestones
    %    Obtain specifications that were out of reach earlier 30
    %    Some theoretical results on complexity of this approach 36
    % Deliverables and contributing staff
    % This will be the topic of the PhD student, advised by HU. We will be able to expand the search space better. All the proposers will provide assistance.

\end{packed_enum}

\noindent\textbf{Present status of the project.} As we noted in the subsection on
feasibility, originality, and impact of our work, our beta has been successful
with several cases. As can be expected from a beta, there are many
obvious avenues for improvements, such as
searching in a more clever manner, looking for combinatorial systems and
description of statistics.

\noindent\textbf{Deliverables.}
Results obtained will be turned into papers and talks at conferences, as is
customary with mathematical research.  We hope that these results will lead to
more work by others in the area. The developed code base will be made publicly
available for others to use and extend.
\\

\spacing{1}
\section{Management and co-operation (domestic/foreign)}
\spacing{1.5}
The proposer has a master degree in Mathematics and Informatics from
Universit\'e du Qu\'ebec \`a Montr\'eal. He has demonstrated a great research
potential by completing original research during his undergraduate and master
programs. The student already has two peer-reviewed publications and a third publication
devoted to permutation patterns under review.
He is also a
good programmer interested in algorithm.
The proposer will be advised by Henning Ulfarsson, an assistant professor
at Reykjavik University. Ulfarsson leads a research group in the field and has
a large network of collaborators.
He has solved two long-standing open
problems in the field: the description of local complete intersection Schubert
varieties, with Woo~\cite{UW11}, and the description of West-$3$-stack-sortable
permutations~\cite{MR2971010}. He co-created the
\bisc{}~\cite{BiSC} and \struct{} algorithms~\cite{algstruct, structpaper}.
He has advised many students on projects in the theory of permutations,
e.g.~\cite{WilfOfShort, parti, MagnussonMSC, MurrayMSC, TomasMSc, BeanPhd}.

The PhD student will also collaborate with Jay Pantone who is an
assistant professor in the Department of Mathematics, Statistics, and Computer
Science at Marquette University.  Pantone was involved in the developments of
the beta version of \motheralg{} and is an expert in analytics combinatorics and
permutations patterns.
\\

\spacing{1}
\section{Proposed publication of results and data storage (including adherence to open access policy)}
\spacing{1.5}
New results will be published in top-tier refereed open access
international journals in discrete mathematics and related fields, as well as
disseminated at seminars and conferences.
\\%
\label{LastPageOfDescription}
\clearpage
\setcounter{page}{1}
\renewcommand{\lastpageref}{\pageref{LastPage}}%

\bibliographystyle{apsrev}

\bibliography{rannis2021}

\end{document}

%%% Local Variables:
%%% mode: latex
%%% TeX-master: t
%%% End:
